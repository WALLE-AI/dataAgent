\title{
3 基本 规 定
}
\section*{3.1 一 般 规 定}
3.1.1 试验环境相对湿度不宜小于 \(50 \%\), 温度应保持在 \(20^{\circ} \mathrm{C} \pm\) \(5^{\circ} \mathrm{C}\) 。
3.1.2 试验仪器设备应具有有效期内的计量检定或校准证书。
\section*{3.2 试件的横截面尺寸}
3.2.1 试件的最小横截面尺寸应根据混凝土中骨料的最大粒径按表 3.2.1 选定。
表 3.2.1 试件的最小横截面尺寸
\begin{tabular}{|c|c|c|}
\hline \multicolumn{2}{|c|}{ 骨料最大粒径 \((\mathrm{mm})\)} & \multirow{2}{*}{\begin{tabular}{c}
试件最小横截面尺寸 \\
\((\mathrm{mm} \times \mathrm{mm})\)
\end{tabular}} \\
\hline 䢃裂抗拉强度试验 & 其他试验 & \\
\hline 19.0 & 31.5 & \(100 \times 100\) \\
\hline 37.5 & 37.5 & \(150 \times 150\) \\
\hline -- & 63.0 & \(200 \times 200\) \\
\hline
\end{tabular}
3.2.2 制作试件应采用符合本标准第 4.1.1 条规定的试模, 并应保证试件的尺寸满足要求。
\section*{3.3 试件的尺寸测量与公差}
3.3.1 试件尺寸测量应符合下列规定:
1 试件的边长和高度宜采用游标卡尺进行测量, 应精确至 \(0.1 \mathrm{~mm}\);
2 圆柱形试件的直径应采用游标卡尺分别在试件的上部、中部和下部相互垂直的两个位置上共测量 6 次, 取测量的算术平均值作为直径 值, 应精确至 \(0.1 \mathrm{~mm}\);